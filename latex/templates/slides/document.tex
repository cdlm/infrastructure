% -*- coding: utf-8; -*-
% \documentclass[%
% 14pt,compress,%
% svgnames,nodvipsnames,%
% notes=show%
% ]{beamer}

\usepackage[english]{babel}
\usepackage{lhelp}
\usepackage[utf8x]{inputenc}
%\usepackage[T1]{fontenc}


\mode<article>{
  \definecolor{linkcolor}{rgb}{0,.2,0}
  \definecolor{urlcolor}{rgb}{0,0,.2}
  \usepackage[pdftex,colorlinks,linkcolor=linkcolor,urlcolor=urlcolor]{hyperref}
  \usepackage{microtype}
  \usepackage[cm]{fullpage}
  \setlength{\columnsep}{3em}\setlength{\columnseprule}{.4pt}
  \raggedright
}

\mode<presentation>{
  \setbeamercolor{structure}{fg=SlateBlue!50!White}
  \setbeamercolor{normal text}{fg=White,bg=DarkSlateBlue!20!Black}
  \setbeamercolor{alerted text}{fg=Tomato!90!normal text.fg}
  \setbeamercolor{example text}{fg=SeaGreen!50!normal text.fg}
  \setbeamerfont{block title}{size=\scriptsize,series=\bfseries}
  \setbeamercolor{block body}{bg=structure.fg!10!normal text.bg}
  \setbeamercolor{block body example}{bg=example text.fg!10!normal text.bg}
  \setbeamerfont{block body}{size=\small}
  \setbeamercolor{note page}{fg=Black}

  \setbeamertemplate{navigation symbols}{}

  %\usetheme{Warsaw}
  \setbeamercovered{transparent}

  \AtBeginSubsection[]{%
    \begin{frame}<beamer>
      \frametitle{Outline}
      \tableofcontents[currentsection,currentsubsection]
    \end{frame}}
  \setbeamertemplate{headline}{\blap{\vskip2ex\hfill\hbox{\insertframenumber\quad}}}
  \setbeamertemplate{footline}{%
    \begin{beamercolorbox}{section in head/foot}%
      \vskip2pt%
      \insertnavigation{\paperwidth}%
      \vskip2pt%
    \end{beamercolorbox}}
  % \beamerdefaultoverlayspecification{<+->}

  \newcommand{\cmd}[1]{{\usebeamercolor[fg]{example text}\texttt{#1}}}
}



\title    {title}
\subtitle {subtitle}
\author   {Damien Pollet}
\institute{Triskell --- Irisa / Université de Rennes 1}
\date[2006]{Where, 2006}



\begin{document}

\mode<article>{
  \maketitle
  \tableofcontents
}

\mode<presentation>{
  \begin{frame}
    \titlepage
  \end{frame}
  
  \begin{frame}
    \frametitle{Outline}
    \tableofcontents
    % You might wish to add the option [pausesections]
  \end{frame}
}


\section{Motivation}

\begin{frame}
  \frametitle{Slide title}
  \note{a note}

  blah blah blah:
  \begin{itemize}[<+->]
  \item foo
  \item bar
  \item quux
  \end{itemize}

  \mode<presentation>{\vfill}
  \visible<+->{\begin{center}
      \alert{Blah!}
    \end{center}}
\end{frame}


\section*{Summary}

\begin{frame}
  \frametitle<presentation>{Summary}
  \begin{itemize}
  \item The \alert{first main message} of your talk, \\
    in one or two lines.
  \item The \alert{second main message} of your talk, \\
    in one or two lines.
  \item Perhaps a \alert{third message}, but \\
    not more than that.
  \end{itemize}
  
  \mode<presentation>{\vskip0pt plus.5fill}
  \begin{itemize}
  \item Outlook
    \begin{itemize}
    \item Something you haven't solved.
    \item Something else you haven't solved.
    \end{itemize}
  \end{itemize}
\end{frame}



\appendix
\section<presentation>*{\appendixname}
\subsection<presentation>*{For Further Reading}

\begin{frame}[allowframebreaks]
  \frametitle<presentation>{For Further Reading}
    
  \begin{thebibliography}{10}
  \bibitem{foo:www} Foo home.
    \newblock \url{http://foo.org}
 
  \bibitem{bar:www} Bar home.
    \newblock \url{http://bar.org}
  \end{thebibliography}
\end{frame}

\end{document}


